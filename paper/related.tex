\section{Related Work}
\label{sec:related}

Most work in the area of smartphone network policies has been in the interest of moving traffic to WiFi \cite{Lee:2010:MDO:1921168.1921203}. This is presumably due to the assumptions stated above and the belief that WiFi is superior to any mobile connection. WiFi connections are obviously favored, as Android will automatically connect to any recognized network it encounters \cite{Google:2013}. The fact that this problem exists is evidence that work done on WiFi connection decisions in smartphone operating systems has simply not considered the issues of pre-mature connection or delayed disconnection from WiFi.

In some sense we could consider efforts to improve mobile networking as working towards our same goal of constant connectivity. Increasing mobile network speed and reliability would reduce a user's reliance on less pervasive WiFi. The latest mobile standards such as LTE have been shown to be faster than home WiFi in some cases \cite{Huang:2012:CEP:2307636.2307658}. While great advances have been made, economic concerns still limit the amount of traffic that can be reasonably transferred over mobile networks,\footnote{Visit any major cell network carrier's website to view the limits and rates they place on mobile data.} ruling out the possibility of eliminating the need for WiFi.