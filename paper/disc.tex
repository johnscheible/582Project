\section{Discussion}
\label{sec:disc}


\subsection{Limitations}

Due to limited (and often nonexistant) documentation of the Android source code, we were forced to make some sacrifices. Optimally, we would have liked to modify Android's network management policies directly. However, the lack of information provided by Google's Android team made this impractical. Therefore, writing an application to manage WiFi at the user level was the best and most feasible option.

This somewhat limited our ability to fully control the mobile device's network management policies because our software ran concurrently with Android's built-in network management. Thus, while we were able to control connectivity, it was impossible to prevent Android from making its own decisions about network connectivity.

Therefore, in cases 1 and 2, our application could not \emph{prevent} connection to unsuitable WiFi networks. Instead, when such a connection occured, we detected the connection and terminated it as fast as Android's framework allows for. In case 3, Android's existing WiFi management policies did not interfere with our policies because we propose to drop connections \emph{before} Android acts.

Additionally, because we only worked with Android (as the next most popular mobile device operating system, iOS, offers no control of WiFi/cell connection handling), it is possible that our data is specific to the domain of Android mobile devices. Other systems may have different WiFi/cell connection management policies.

Additionally, different mobile devices may require different policies. We tested our policies on a single phone, but other devices may require adjustments such as altered time or signal strength thresholds. Also, devices capable of more accurately sensing signal strength are inherently better suited to making decisions based on such data.

Another limitation of our research is that 
