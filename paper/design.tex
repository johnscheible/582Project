\section{Design}
\label{sec:design}

% SPECIFICALLY address each of the 3 problems presented in the intro. Show how one policy fixes 2 of the problems, and how the other fixes 1.

\subsection{Internet Access Testing} % TODO - ask John about this. I did a pretty small, crappy writeup
In order to address problems 1 and 2, we designed a policy that would ensure that a mobile device does not connect to a WiFi network until it knows it will be capable of reaching the Internet over said network. The test is simple: when a new WiFi network is encountered and the device is using cellular data, attempt to ping a well known website (in our case, umich.edu) via the new WiFi network. The test is repeated a certain number of times in order to ensure that the WiFi network is offering the mobile device a stable connection.

If the correct server responds to our GET request with a 200 response (indicating a successful request), the thread sleeps for a set amount of time (in our case, 3 seconds) and continues pinging until we hit the number of tests we want to conduct (in our case, 2).

If any of these tests fails, we terminate the WiFi connection, and continue to use the cellular connection. If all the tests succeed, the device is switched over to the WiFi network. This, optimally, will result in no connection downtime whatsoever, as the WiFi connection is not utilized until it has been tested for Internet accessibility.

The reason that access is tested multiple times over a time interval is to establish reasonable evidence that the connection will remain strong after testing is concluded. Were the connection only tested once, it is very easy to imagine that a connection which will only be reachable for a couple of seconds (i.e. a scenario where the device briefly skirts the network's periphery) could be deemed valid and thus connected to.

The system is activated whenever the device is utilizing its cellular data connection, and encounters a new WiFi network.\footnote{We did consider the possibility of testing each individual access point (rather than WiFi network) encountered, but this would result in a massive number of WiFi scans that would considerably drain the device's battery.}

\subsection{Predictive WiFi Abandonment}
In order to address problem 3, we designed a policy that would preemptively terminate connections that are likely to degrade to the point of unusability. Currently, a mobile Android device will not disconnect from a WiFi network until it knows that it cannot reach the network anymore. This leads to downtime while the device has not yet determined that reaching the network is impossible, but traffic is not reaching its destination. The idea of our policy is based on a simple concept: that such downtime can be prevented by predicting when the network will become unreachable, not when it already has become so.

The test itself monitors the strength of a device's current WiFi connection. This information is obtained by querying the device's WiFi radio, which returns a received signal strength indication (RSSI) value. The range of possible RSSI values varies greatly depending on the device in question (i.e. some devices return vales from 0 to 100, some return values from -200 to 0, etc.), and therefore each device must normalize the RSSI values in order to make decisions.

In order to predict when WiFi quality is degrading to the point that it is better to switch to cellular data than remain connected to WiFi, the device monitors the signal strength over time. When the signal strength drops below a certain threshold for a certain amount of time (on our device, -80 RSSI units for 6.9 seconds), the system disconnects from WiFi and switches to cellular data. Ideally, this switch will result in no outgoing packet loss, as outgoing traffic is immediately diverted from WiFi to the cellular data network.

The system is active whenever WiFi is being used, and runs as a background process.