\section{Introduction}
\label{sec:intro}

In the era of always-connected mobile devices, a new breed of application has arisen that depends on a constant connection between mobile devices and the Internet. Users have come to rely heavily on the always-connected property of these applications for regular and time-sensitive information. Applications that utilize always-on connections generally fall into these categories:

\begin{description}
\item[Notification-based] Applications relying on ``push notifications" to alert users to pertinent, time-sensitive information. This class of applications includes email clients alerting users to a new email, social applications (e.g. Facebook, social games) updating users on relevant events, messaging applications (e.g. iMessage, KaokaoTalk), applications capable of providing emergency alerts, and other services that must present information as it becomes available.
\item[Streaming] Applications streaming data that rely on constant Internet connectivity to deliver uninterrupted content. Common types of streaming applications include audio streaming (e.g. Pandora, Spotify), video streaming (e.g. Youtube, Netflix), and real-time multiplayer games. Voice-over-IP (VOIP) services are also becoming more common on mobile devices (e.g. Skype, KaokaoTalk). Major network operators currently have plans to transition their entire networks to ``IP-only" \cite{ATT:2012} moving all voice and messaging services to connectivity-reliant applications.
\end{description}

Always-connected applications require guarantees of Internet connectivity. These applications do not function correctly when they suffer from intermittent or prolonged downtime.

Notification-based applications can likely tolerate a few seconds of downtime, although it is easy to imagine scenarios in which an application cannot tolerate downtimes on the order of minutes. For instance, having a conversation via instant messaging becomes difficult, and not very ``instant" when Internet connectivity is lost.

Streaming applications, however, cannot tolerate such downtime. \emph{Real-time} streaming applications such as VOIP that cannot buffer for more than a few seconds absolutely cannot tolerate it. VOIP is impractical when the connection is dropped for more than a few seconds at a time. Applications that can buffer data, such as streaming video and audio applications, can tolerate some downtime, although in many cases this is limited to a number of seconds.

For these and other reasons, mobile devices aim to provide 100\% Internet connectivity uptime. Connections are either established over a cellular data network, such as GSM or LTE, or a WiFi connection. Modern smartphone operating systems will connect to any recognized WiFi network when in range of it \cite{Google2013}. This policy is presumably based on a set of assumptions:

\begin{itemize}
\item WiFi connections are always faster than cellular connections.
\item Stable WiFi connections are more reliable than stable cellular connections.
\item Customers are charged for their cellular connections by the amount of data used.
\end{itemize}

While these may have been true at one time, they are becoming less and less so every day. With the advent of widely-available technologies like 4G LTE, the gap between WiFi and cellular connection speed has been shrinking \cite{Huang:2012:CEP:2307636.2307658}. While data caps are a legitimate concern, if no data is preferable to mobile data, modern smartphone operating systems include the option to disable it. Therefore, these priorities should no longer be the only factors in choosing whether to connect to the Internet via WiFi or a cellular data network.

The ``WiFi first and always" policy creates serious problems for always-connected applications. In many everyday scenarios, Internet connectivity is lost while mobile devices cling to unusable or weak WiFi networks. In these scenarios, Internet connectivity could be maintained by utilizing a cellular data connection, but the device instead loses connectivity because it tries to connect to an unusable WiFi network.

Three specific instances of this problem stand out, as they affect the average mobile device user (this paper's authors included) by regularly disconnecting the device from the Internet when a cellular connection could be used:

\begin{enumerate}
\item The mobile device connects to a WiFi network that is too weak to send data over.
\item The mobile device connects to a WiFi network that requires authentication before the Internet can be accessed.
\item The mobile device remains connected to a WiFi network after the network becomes too weak to send data over.
\end{enumerate}

In all of these cases, mobile devices running the Android or iOS operating systems will often remain disconnected from the Internet until the user manually intervenes by forcing the device to abandon WiFi (usually by disabling the device's WiFi feature entirely). These problems can result in downtime anywhere between less than a second to as long as it takes for a user to recognize and resolve the issue affecting the mobile device. In each of these scenarios, Internet connectivity could be maintained by utilizing the cellular connection. The policies we propose to resolve these problems are:

\begin{description}
\item[Internet Access Testing] In order to address problems 1 and 2, we propose to test and judge the usability of WiFi networks before making decisions about whether to connect to them. The goal of this policy is to make informed decisions about whether connecting to a WiFi network can be done seamlessly (i.e. while maintaining connectivity) or whether it will result in loss of Internet connectivity.
\item[Predictive WiFi Abandonment] In order to address problem 3, we propose to periodically sample the strength of the WiFi network the mobile device is currently connected to. Using this data, we can predict whether a WiFi connection will become too weak to function, and switch to a cellular data connection before Internet connectivity is lost.
\end{description}

